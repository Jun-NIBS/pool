\subsection{Improvement for Searching for Specific Activity} \label{sec:extension}
In our application to searching for peptides with specific activity, we do not simply obtain a binary activity value $y(e)$
for each peptide tested. Instead, we observe multiple binary values $y_i(e)$, each associated with a reaction between the 
peptide and a specific enzyme, and we determine the final activity as a composition of these observed binary values. 
For example, when searching for \enquote{Sfp-specific hits}, the binary value indicating this activity is the product of 
three observable binary values: the value that is 1 if the peptide has activity for Sfp and 0 if not; the value that is 0 
if the peptide has activity for AcpS and 1 if not; and the value that is 1 if the peptide is unlabeled by ACPH and 0 if not.
Mathematically, the activity of peptide $e$ is $y(e) = \prod_i y_i(e)$. We could use POOL as described above, where we train 
our Naive Bayes classifier only on $y(e)$, however, the peptides that are not tested for all three enzymes will be wasted in
this approach because we cannot obtain $y(e)$ for those peptides. For example, in our application, a fraction of the tested 
peptides ($\sim 300$) were only tested for Sfp activity and ACPH activity, where we do not have information on whether they 
are \enquote{Sfp-specific hits} or \enquote{AcpS-specific hits}. To utilize as much information as possible, instead of building 
Naive Bayes classifier that predicts $y(e)$ directly, we build separate Naive Bayes classifiers that predict each $y_i(e)$, and rewrite \eqref{eq:reduced form} as
\begin{equation}
  \underset{e \in E \backslash S, f(e) > b}{\mathrm{arg}\max} \, \Prob (y_i(e)=1, \forall i \mid \prod_i y_i(e') = 0, \forall e' \in S).
  \label{eq:extension}
\end{equation}
This objective is hard to compute because the number of configurations of $y_i(e'), \forall e \in S$ that satisfy $\prod_i y_i(e') = 0$ grows exponentially with the size of $S$, and the computation quickly becomes infeasible as size of $S$ grows. We adopt a heuristic approach, which is to approximate the objective by
\begin{equation}
  \Prob \left( y_i(e)=1, \forall i \mid y_i(e'), \forall i, \forall e' \in S \right),
  \label{}
\end{equation}
and we choose $y_i(e')$ as long as they satisfy $\prod_i y_i(e') = 0, \forall e' \in S$.