\documentclass[12pt,fleqn]{article}
\usepackage{amsmath,amssymb,mathtools,amsthm}
\usepackage{setspace}
\usepackage[margin=1in]{geometry}
\setlength{\parindent}{0pt}

\newcommand{\EI}{\text{EI}}
\newcommand{\PI}{\text{P}^*}

\newcommand{\mb}{\mathbf}

\begin{document}

\section{Problem State}
Let $E$ be any set. For each element $x\in E$, we define a function $f(x)$ that measures the quality of $x$. Larger or smaller values of $f(x)$ may be favored depending on different problem settings. From now on, we assume, without losing generality that smaller values of $f(x)$ are preferred, and for any subset $S\subseteq E$, we measure the quality of $S$ as:

\begin{equation*}
f^*(S) = \min_{x\in S:\mathbf{h}(x)=0}f(x)
\end{equation*}

where $\mathbf{h}(x)=(h_1(x),\cdots,h_m(x))$ is a set of constraints that define a subset of "effective elements". We wish to find $S\subseteq E$ with $f^*(S)$ as small as possible, while $S$ it self must satisfy some constraints. A typical constraint is the cardinality of $S$, we usually prefer smaller sets. Other constraints can be applied in different problems.

Let $b$ be a target value and we wish to find $S\subseteq E$ such that $f^*(S)$ is, in some sense, better than $b$. Specifically, we consider the following two measures:

\begin{eqnarray*}
&\text{Probability of Improvement: }&\PI(S) = P(f^*(S) < b)\\
&\text{Expected Improvement: }&\EI(S) = E[(b-f^*(S))]
\end{eqnarray*}

We wish to find $S$ that maximize one of these two measures. Let $g(S)$ be either $\PI(S)$ or $\EI(S)$ and let the cardinality of $S$ be the only constraint on $S$. Our goal is then:

\begin{equation*}
\max_{S\subseteq E:|S|<k}g(S)
\end{equation*}

\end{document}